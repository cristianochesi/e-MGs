\documentclass[10pt,twoside,a4paper]{memoir}
\usepackage{graphicx}
\usepackage{forest}
\usepackage{tikz-qtree}
\usetikzlibrary{arrows.meta}
\usepackage{geometry}
\geometry{a4paper, portrait, margin=1in}

% define move style
\tikzset{move/.style = {-{Latex[length=.5em]},dashed,blue}}
\tikzset{rightward/.style = {dotted}}
%\tikzset{annotation/.style = {font=\footnotesize}}
\tikzset{annotation/.style = {font=\footnotesize,inner sep=2.5pt,outer sep=2.5pt}}
\tikzset{index/.style = {annotation, anchor=south east}}
\tikzset{outdex/.style = {annotation, anchor=north west}}
\tikzset{boxed/.style = {draw}}
\tikzset{empty/.style = {}}
\tikzset{non-final/.style = {opacity=70}}

% indexed node labels in trees;
% these macros are only needed if you want to use the output of the .fprint method
\newcommand{\Lab}[3]{\tsp{#2}#1\tsb{#3}} % for normal nodes
\newcommand{\BLab}[3]{\tsp{#2}#1{\setlength{\fboxsep}{.25\fboxsep}\boxed{\tsb{#3}}}} % for boxed leaves
\newcommand{\IBLab}[3]{\tsp{#2}#1{\setlength{\fboxsep}{.25\fboxsep}\boxed{\tsb{#3}}}} % for boxed interior nodes

\begin{document}

\begin{center}

% substitute below the tree-code generated by the script
\begin{forest}
[ROOT, name=0
	[I , name=2]
	[saw, name=5
		[$t_{2}$ , name=7]
		[the, name=8
			[giraffe, name=10
				[that, name=12
					[the, name=15
						[cow , name=17]
					]
					[kicked, name=21
						[$t_{15}$ , name=23]
						[$t_{10}$ , name=24]
					]
				]
			]
		]
	]
]
\draw[move = {canonical}] (2) to[out = south west, in = south west] (7);
\draw[move = {canonical}] (15) to[out = south west, in = south west] (23);
\draw[move = {canonical}] (10) to[out = south west, in = south west] (24);
%
\node[index] at (0) {0};
\node[outdex] at (0) {5};
%
\node[index] at (2) {2};
\node[outdex] at (2) {2};
%
\node[index] at (5) {5};
\node[outdex] at (5) {8};
%
\node[index] at (7) {2};
\node[outdex] at (7) {7};
%
\node[index] at (8) {8};
\node[outdex] at (8) {10};
%
\node[index] at (10) {10};
\node[outdex] at (10) {12};
%
\node[index] at (12) {12};
\node[outdex] at (12) {21};
%
\node[index] at (15) {15};
\node[outdex] at (15) {17};
%
\node[index] at (17) {17};
\node[outdex] at (17) {17};
%
\node[index] at (21) {21};
\node[outdex] at (21) {21};
%
\node[index] at (23) {15};
\node[outdex] at (23) {23};
%
\node[index] at (24) {10};
\node[outdex] at (24) {24};
\end{forest}
% end of the tree-code to be substituted

\end{center}
\end{document}